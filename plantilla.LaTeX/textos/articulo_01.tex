%!TEX root = ../index.tex
% - - - - - - - - - - - - - - - - - - - - - - - - - - - - - - - - - - - -

\capitulo{El comienzo}

Test para las notas\footnote{test} \footnote{test}

Ver el archivo \enquote{lorem.tex} para notas sobre la selección de tipografías. Mac y Linux emplean \enquote{xelatex}, por lo que no hace falta definir codificaciones \textbf{T1} o similares.

Esta línea es para que se ver la bibliografía funcionando... como aparece en... \cite{libro1}.

Esta otra está para probar la palabra del índice. \index[p]{Prueba normal}

Y finalmente esta está para probar el índice onomástico. \index[g]{Persona}

\blindtext[2]