%!TEX root = ../index.tex
% - - - - - - - - - - - - - - - - - - - - - - - - - - - - - - - - - - - -

%%!TEX root = ../index.tex
% - - - - - - - - - - - - - - - - - - - - - - - - - - - - - - - - - - - -

% demo de cambio de tipografía

Cambiamos a la tipografía definida con:

\begin{verbatim}
    \medieval Texto de prueba
\end{verbatim}

Y obtenemos:
\medieval Texto de prueba.

\rmfamily
La tipografías llamas {\medieval medieval} se ha quedado activada, pero la hemos devuelto con: 

\begin{verbatim}
    \rmfamily
\end{verbatim}


Así que:

\blindtext[1-2]

\capitulo{El comienzo}

Ver el archivo \enquote{lorem.tex} para notas sobre la selección de tipografías. Mac y Linux emplean \enquote{xelatex}, por lo que no hace falta definir codificaciones \textbf{T1} o similares.

Esta línea es para que se ver la bibliografía funcionando... como aparece en... \cite{libro1}.

\blindtext[2]

\capitulo{La continuación}
\Blindtext

\capitulo{El último}
\Blindtext